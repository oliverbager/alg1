\section{Groups}
\subsection{Definition}
A composition on a set $G$ is a map $\circ:G\times G\rightarrow G$. The composition $\circ(g,h)$ is often denoted $g\circ h$ or $gh$.
\begin{defi}[Group]
    A pair $(G,\circ)$ consisting ofa set $G$ and a composition $\circ:G\times G\rightarrow G$ is called a group if it satisfies:
    \begin{itemize}
        \item[(i)] $\forall s_{1},s_{2},s_{3}\in G,s\circ(s_{2}\circ s_{3})=(s\circ s_{2})\circ s_{3}$, the composition is associative.
        \item[(ii)] For some $e\in G$, $e\circ s=s$ and $s\circ e=s$, there exists a neutral element.
        \item[(ii)] For every $s\in G$ there exists an inverse element, $t$, such that $t\circ s=e$ and $s\circ t=e$.
    \end{itemize}
    A group is called abelian if $x\circ y=y\circ x$ for every $x,y\in G$, whilst the number of elements $|G|\in G$ is called the order of $G$.
\end{defi}
An example of a non-abelian group is $(\mathbb{N},+)$.
\begin{exmp}
    The neutral element for $(\mathbb{N},+)$ would have to be 0, except a problem then occurs as no inverse element would exist (due to us not having negative numbers), as no solution $\in\mathbb{N}$ would exist for $n+t=0$ where $n,t\in\mathbb{N}$.
\end{exmp} 
To exemplify an abelian group we examine $(\mathbb{Z},+)$.
\begin{exmp}
    The neutral element for $(\mathbb{Z},+)$ would again have to be 0, but this time we have a well defined inverse element as the equation $n+t=0$ has solution $t=-n$, both of which being contained in $\mathbb{Z}$, additionally it is abelian as $n+t=t+n$.
\end{exmp} 
Similarly we can look at larger sets such as the rationals and reals, which are abelian in regards to addition, if we look at multiplication instead, we have to exclude 0 from the groups as 1 would have to be the neutral element by (ii), but then no value $t$ exists such that $0\times t=1$.
\subsection{Groups and congruences}
We often times work with $\mathbb{Z}$ instead of $\mathbb{N}$ as we then have an inverse. Given that $\mathbb{Z}$ is a group with composition +, we can build new groups tied to the congruence modulo some integer. We define addition on a subset of $\mathbb{Z}$ given by $a+n\mathbb{Z}=\{a+nx~|~n\in\mathbb{Z}\}$ where $a,n\in\mathbb{Z}$.
\begin{exmp}
    Let $G=2+3\mathbb{Z}$ and $H=5+3\mathbb{Z}$, then
    \begin{align*}
        G&=\{\ldots,-4,-1,2,5,\ldots\} \\
        H&=\{\ldots,-1,2,5,8,\ldots\}
    \end{align*}\vskip -10pt
\end{exmp}
We quickly identify that the two sets contain the same elements, and as such we say they are identical.
\begin{prop}[Identical sets]
    Let $a,b,c\in\mathbb{Z}$ then: 
    \begin{itemize}
        \item[(i)] $a+c\mathbb{Z}=b+c\mathbb{Z}\iff a\equiv b\mod c$.
        \item[(ii)] $(a+c\mathbb{Z})\cap(b+c\mathbb{Z})=\emptyset\iff a\not\equiv b\mod c$.
    \end{itemize}
\end{prop}
\begin{prf}
    Let $m\in a+c\mathbb{Z}$, then if $a+c\mathbb{Z}=b+c\mathbb{Z}\implies m\in b+c\mathbb{Z}$. As such for some $x,y\in\mathbb{Z}$ we have $m=a+cx=b+cy\implies a-b=c(y-x)$ meaning that $a\equiv b\mod c$, whereby $a=b+cx\implies a+c\mathbb{Z}=b+cx+c\mathbb{Z}=b+c\mathbb{Z}$ as $x+c\in \mathbb{Z}$, proving (i). Now if $(a+c\mathbb{Z})\cap(b+c\mathbb{Z})\neq\emptyset$ there must exists elements $m,x,y\in\mathbb{Z}$ such that $m=a+cx=b+cy\implies a-b=c(y-x)$ whereby $a\equiv b\mod c$ by (i).
\end{prf}
If $c>0$ then $a+c\mathbb{Z}=b+c\mathbb{Z}\iff [a]_{c}=[b]_{c}$. Let $[x]$ be the subset $x+c\mathbb{Z}$, being the set consisting of integers giving remainder $x$ following divison by $c$.
\begin{exmp}
    Let $c=3$ then $\mathbb{Z}/3\mathbb{Z}=\{[0],[1],[2]\}$, where
    \begin{align*}
        [0]=\{\ldots,-6,-3,0,3,6,\ldots\} \\
        [1]=\{\ldots,-5,-2,1,4,7,\ldots\} \\
        [2]=\{\ldots,-4,-1,2,5,8,\ldots\}
    \end{align*}\vskip -10pt
\end{exmp}
Using this we can add subsets $[x],[y]\in\mathbb{Z}/c\mathbb{Z}$ by defining $[x]+[y]=[x+y]$, as we by Proposition 1.3.2(i) have that if $[x]\equiv [x']$ and $[y]\equiv [y']$ then $[x+y]=[x'+y']$, using this we check whether $(\mathbb{Z}/c\mathbb{Z},+)$ forms a group. Since associativity holds for $(\mathbb{Z},+)$, we have that
\begin{align*}
    &([x]+[y])+[z]=[x+y]+[z]=[(x+y)+z] \\
    &\qquad=[x+(y+z)]=[x]+[y+z]=[x]+([y]+[z])
\end{align*}
similarly, we have that 0 is the neutral element as
\[
    [x]+[0]=[x+0]=[x]
\]
whilst there also exists an inverse by $-x$ as
\[
    [x]+[-x]=[x-x]=[0]
.\]
This group is also abelian as $[x]+[y]=[x+y]=[y+x]=[y]+[x]$. In the case that $c$ is 0 then $\mathbb{Z}/0\mathbb{Z}=\{x+0\mathbb{Z}\}=\{x\}=\mathbb{Z}$, where addition is obviously defined.
\subsubsection{Composition table}
\begin{defi}[Composition table]
    Given a finite group $\{e,g_{1},\ldots,g_{r}\},\cdot)$ the composition is often shown using a composition table:

    \centering{\begin{array}{c|cccccc}
        $\circ & e & g_{1} & \cdots & g_{j} & \cdots & g_{r} \\
        \hline
        e & e & g_{1} & \cdots & g_{j} & \cdots & g_{r} \\
        g_1 & g_{1} & g_{1}\circ g_{1} & \cdots & g_{1}\circ g_{j} & \cdots & g_{1}\circ g_{r} \\
        \vdots & \vdots & \vdots & \ddots & \vdots & \ddots & \vdots \\
        g_{i} & g_{i} & g_{i}\circ g_{1} & \cdots & g_{i}\circ g_{j} & \cdots & g_{i}\circ g_{r} \\
        \vdots & \vdots & \vdots & \ddots & \vdots & \ddots & \vdots \\
        g_{r} & g_{r} & g_{r}\circ g_{1} & \cdots & g_{r}\circ g_{j} & \cdots & g_{r}\circ g_{r}$
     \end{array}}
\end{defi}
\begin{exmp}
  The composition table for the previously computed group ($\mathbb{Z}/3\mathbb{Z},+)$ is given by
  \begin{center}
  \begin{array}{c|ccc}
          $+ & [0] & [1] & [2] \\
          \hline
          [0] & [0] & [1] & [2] \\
          \[[1] & [1] & [2] & [0] \\
          \[[2] & [2] & [0] & [1]$
  \end{array}
  \end{center}
\end{exmp}
\subsection{Associativity}
Suppose that $S$ is a set with a multiplicative group mapping $(x,y)\rightarrow xy$. As it is a group, associativity holds, an expression like $s_{1}s_{2}\ldots s_{n}$ for $n>2$ isnt valid for this composition, as it only takes in two element, however we can reduce the expression to contain only two elements by repeatedly multiplying sets of two elements.
\begin{exmp}
  Consider the sequence $s_{1}s_{2}s_{3}s_{4}$ in the previous composition, we can calculate this in 5 different ways
  \begin{align*}
    &s_{1}(s_{2}(s_{3}s_{4})) \\
    &s_{1}((s_{2}s_{3})s_{4}) \\
    &(s_{1}(s_{2}s_{3}))s_{4} \\
    &((s_{1}s_{2})s_{3})s_{4} \\
    &(s_{1}s_{2})(s_{3}s_{4})
  \end{align*}
\end{exmp}
Determining whether at set is associative is difficult, except for one case: when $S$ is the set of maps from a set $X$ to itself where composition is defined as usual, meaning that $fg$ is $(fg)(x)=f(g(x))$ for $f,g\in S$ and $x\in X$. In which case $f(gh)=(fg)h$ since $(f(fg))(x)=((fg)h)(x)$ for all $x\in X$:
\begin{align*}
    (f(gh))(x)&=f((gh)(x))=f(g(h(x))) \\
    (f(g)h)(x)&=(fg)(h(x))=f(g(h(x)))
\end{align*}
\subsection{Example of non-abelian group}
\begin{exmp}
    Let $X=\{1,2,3\}$ and $G$ be the set of all bijective maps $X\rightarrow X$, then $G$ is a group with the usual composition of maps as composition. Here the neutral element of $G$ will be the identity map $X\rightarrow X$, the inverse of some map $f:X\rightarrow X$ is the inverse map $f^{-1}:X\rightarrow X$, and the composition of maps is associative, we can list the elements of $G$ as
    \begin{align*}
        e&=\begin{pmatrix}1 & 2 & 3 \\ 1 & 2 & 3\end{pmatrix}\hskip 16pt a=\begin{pmatrix}1 & 2 & 3 \\ 2 & 1 & 3\end{pmatrix}\hskip 16pt b=\begin{pmatrix}1 & 2 & 2 \\ 1 & 3 & 2\end{pmatrix} \\
        c&=\begin{pmatrix}1 & 2 & 3 \\ 3 & 2 & 1\end{pmatrix}\hskip 16pt d=\begin{pmatrix}1 & 2 & 3 \\ 3 & 1 & 2\end{pmatrix}\hskip 16pt f=\begin{pmatrix}1 & 2 & 3 \\ 2 & 3 & 1\end{pmatrix}
    \end{align*}
    where for example $c:X\rightarrow X$ is the map given by $c(1)=3,c(2)=2,c(3)=1$. Compositions are simple to find, as we can apply associativity, finding for example that $ab(x)=a(b(x))$ implying that $a(b(1))=a(1)=2,a(b(2))=a(3)=3,a(b(3))=a(2)=1$, also showing that $ab=f$. The composition table is given by
    \begin{center}
    \begin{array}{c|cccccc}
        $\circ & e & a & b & c & d & f \\ 
        \hline
        e & e & a & b & c & d & f \\ 
        a & a & e & f & d & c & b \\
        b & b & d & e & f & a & c \\
        c & c & f & d & e & b & a \\
        d & d & b & c & a & f & e \\
        f & f & c & a & b & e & d$
    \end{array}
    \end{center}
    This group is known as the symmetric group $S_{3}$ and is non-abelian, as $ab\neq ba$.
\end{exmp}
\subsection{Uniqueness of neutral and inverse}
Suppose that $e,e'\in G$ are neutral elements of $G$, then $e=e'e=e'$, and as such $e'=e$. Similarly to every $g\in G$ there can only be one inverse element $g^{-1}$, let $g^{-1}'$ be a second inverse element satisfying $g^{-1}'g=e$, then $g^{-1}'g=e=g^{-1}g$, then by multiplying both sides by $g^{-1}'$ we have that $g^{-1}'=(g^{-1}g)g^{-1}'=g^{-1}(gg^{-1})=g^{-1}$.
\begin{defi}[Inverse element]
    Let $g\in G$ be an element of a group, then we let $g^{-1}\in G$ denote the unique inverse element of $g$.
\end{defi}
\subsection{Multiplication by $g\in G$ is bijective}
Let $G$ be a group and $g\in G$, then a map $\varphi:G\rightarrow G$ given by $\varphi(x)=gx$ is bijective, we prove this by providing the inverse map $\lambda:G\rightarrow G$ given by $\lambda(x)=g^{-1}x$. Then $\lambda(\varphi(x))=g^{-1}(gx)=(g^{-1}g)x=ex=x$. proving that $\lambda=\varphi^{-1}$, as the inverse exists, $\varphi$ must be bijective.
\subsection{Subgroups and cosets}
\begin{defi}[Subgroup]
    A subgroup of a group $G$ is a non-empty subset $H\subseteq G$ such that the composition of $G$ makes $H$ into a group, i.e., $H$ is a subgroup of $G$ if and only if:
    \begin{itemize}
        \item[(i)] $e\in H$.
        \item[(ii)] $x^{-1}\in H$ for all $x\in H$.
        \item[(iii)] $xy\in H$ for all $x,y\in H$.
    \end{itemize}
\end{defi}
\begin{exmp}
    Observe group $S_{3}$ from Example 2.4.1, here $\{e,a\}$ and $\{e,f,d\}$ are subgroups of $S_{3}$ by the composition table.
    \begin{center}
    \begin{array}{c|cccccc}
        $\circ & e & d & f \\ 
        \hline
        e & e & d & f \\ 
        d & d & f & e \\
        f & f & e & d$
    \end{array}\hskip 32pt
    \begin{array}{c|cccccc}
        $\circ & e & a \\ 
        \hline
        e & e & a \\ 
        a & a & e$ \\
    \end{array}
    \end{center}
\end{exmp} 
We recall from previous chapters that $(\mathbb{Z},+)$ forms a group, in the context of groups the uniqueness of the remainder following divison results in
\begin{prop}[Subgroups of $\mathbb{Z}$]
    Let $H$ be a subgroup of $(\mathbb{Z},+)$ then:
    \[
        H=d\mathbb{Z}=\{dn~|~n\in \mathbb{Z}\}=\{\ldots -2d,-d,0,d,2d,\ldots\}
    \]
    for a unique $d\in\mathbb{N}$.
\end{prop}
\begin{prf}
    If $d=0\implies H=\{0\}$ Assume $H\neq\{0\}$, then $N\cap H$ contains a smallest natural number $d>0$ as the first positive non-zero entry is always $d$ given that $d\neq 0$. Let $H=d\mathbb{Z}$, then $-d\in H$ by definition as $d\in H$, $H$ being a subgroup. Similarly, the set must be closed under addition, and as such $d+d,d+d+d,d+d+d+d,\ldots\in H$ (and the same for negative $d$), showing that all $nd\in H$ for $n\in\mathbb{Z}$, whereby $d\mathbb{Z}\subseteq H$. Suppose now that $m\in H$, division with remainder gives that $m=qd+r$ where $0\leq r<d$. Since $H$ is a subgroup, $m,d,-qd,r\in H$, but as $d>r\in\mathbb{N}\geq 0$, $r$ can only be 0 whereby $m=qd\implies H=d\mathbb{Z}$.
\end{prf}
Let $H$ be a subgroup of $G$ and $g\in G$, then the subsets
\[
    gH=\{gh~|~h\in H\}\subseteq G\hskip 32pt Hg=\{hg~|~h\in H\}\subseteq G
\]
Are called the left- and right cosets respectively. The set of left cosets of $H$ is denoted $G/H$ whilst the set of right cosets of $H$ is denoted $H\setminus G$.
\begin{exmp}
    Let $G=(\mathbb{Z},+),H=3\mathbb{Z}$, then
    \[
        \mathbb{Z}/3\mathbb{Z}=\{3\mathbb{Z},1+3\mathbb{Z},2+3\mathbb{Z}\}
    \]
    As we are working modulo 3 the sets $1+3\mathbb{Z}=4+3\mathbb{Z}=1+3n+3\mathbb{Z}$ for $n\in\mathbb{Z}$, as such two subgroups of $H$ of $G$, $g_{1}H=g_{2}H$ does not necessarily imply that $g_{1}=g_{2}$.
\end{exmp}
\begin{exmp}
    Let $H=\{e,a\}\suset S_{3}$, then all left- and right cosets of $H$ can be listed using the composition table as
    \begin{align*}
        eH=\{ee,ea\}&=\{e,a\}\hskip 32pt He=\{ee,ae\}=\{e,a\} \\
        aH=\{ae,aa\}&=\{a,e\}\hskip 32pt Ha=\{ea,aa\}=\{a,e\} \\
        bH=\{be,ba\}&=\{b,d\}\hskip 32pt Hb=\{eb,ab\}=\{b,f\} \\
        cH=\{ce,ca\}&=\{c,f\}\hskip 32pt Hc=\{ec,ac\}=\{c,d\} \\
        dH=\{de,da\}&=\{d,b\}\hskip 32pt Hd=\{ed,ad\}=\{d,c\} \\
        fH=\{fe,fa\}&=\{f,c\}\hskip 32pt Hf=\{ef,af\}=\{f,b\}
    \end{align*}
    We quickly recognize that these are either equal or disjoint, for the right cosets we have $H=eH=aH,bH=dH,cH=fH$ whilst for the left cosets $H=He=Ha,Hb=Hd,Hc=Hf$, meaning that we can write the sets of cosets as $G/H=\{H,bH,cH\}$ and $G\setminus H=\{H,Hb,Hc\}$.
\end{exmp}
\pagebreak\begin{lemm}
  Let $H$ be a subgroup of a group $G$ and let $x,y\in G$, then:
  \begin{itemize}
      \item[(i)] $x\in xH$.
      \item[(ii)] $xH=yH\iff x^{-1}y\in H$.
      \item[(iii)] If $xH\neq uH$ then $xH\cap yH=\emptyset$.
      \item[(iv)] The map $\varphi:H\rightarrow xH$ given by $\varphi(h)=xh$ is bijective.
  \end{itemize}
\end{lemm}
\begin{prf}
    $x$ is obviously contained in $xH$ as $x=xe$, proving (i). If $xH=yH$ then there exists an $h\in H$ such that $xh=ye=y$, making $y=xh$ and thereby $yH\subseteq xH$, similarly, by the previous equality it follows that $x=yh^{-1}$ whereby we also have $xH\subseteq xH$, which is satisfied if and only if $xH=yH$, proving (ii). Now let $z\in xH\cap yH$, then $z=xh_{1}=yh_{2}$, for some $h_{1},h_{2}\in H$, then there must exist $h_{1}$ such that $xh_{1}=yh_{2}\implies x^{-1}y\in H$, which by (ii) shows that $xH=xY$, whereby (iii) holds. We let $\lambda(h)=x^{-1}h$, whereby $\lambda(\varphi(h))=x^{-1}xh=(x^{-1}x)h=eh=h$, proving that the inverse exists whereby it must be bijective.
\end{prf}
\begin{coro}
    Let $H$ be a subgroup of $G$, then
    \[
        G=\bigcup_{g\in G}gH
    \]
    and if $g_{1}H\neq g_{2}H$, then $g_{1}H\cap g_{2}H=\emptyset$.
\end{coro}
\begin{prf}
    By the proposition we have that $g_{1}H\neq g_{2}H$, as such by Lemma 2.7.1(iii) their intersection is the empty set.
\end{prf}
\begin{theo}[Lagrange]
    If $H\subseteq G$ is a subgroup of a fionite group $G$, then
    \[
        |G|=|G/H||H|
    \]
    Meaning that the order of a subgroup divides the order of the group.
\end{theo}
\begin{prf}
    Let $gH$ be a coset in $G/H$, by Lemma 2.7.1(iv) the map $\varphi:G\rightarrow gH$ is bijective, and as such $|H|=|gH|$. Now as $G=\bigcup_{i\in G} g_{i}H$, the order of $|G|$ must be the order of $|H|$ multiplied by the amount of cosets, whereby $G=|G/H||H|$ implying that $|H|$ divides $|G|$.
\end{prf}
\begin{defi}[Index]
    The number of cosets $|G/H|$ is called the index of $H\in G$, denoted $[G:H]$.
\end{defi}
